\documentclass[dvipdfmx]{jsarticle}
\usepackage[T1]{fontenc}
\usepackage[dvipdfmx]{hyperref}
\usepackage{lmodern}
\usepackage{latexsym}
\usepackage{amsfonts}
\usepackage{amssymb}
\usepackage{mathtools}
\usepackage{amsthm}
\usepackage{multirow}
\usepackage{graphicx}
\usepackage{wrapfig}
\usepackage{here}
\usepackage{float}
\usepackage{ascmac}
\usepackage{url}

\title{人物相関図の可視化-課題研究3-}
\author{文理学部情報科学科\\5419045 高林 秀}
\date{\today}

\begin{document}

\maketitle

\begin{abstract}
本稿は、今年度コンピューティング2のネットワークの可視化に関する課題として、小説レ・ミゼラブルの人物相関図をグラフで可視化する実験を行うものである。なお、本課題ではp5.jsを使用した。
\end{abstract}

\section{目的}
本稿は、今年度コンピューティング2の課題研究として「ネットワークの可視化」に関する問題に解答するものである。また同時に、問題に関する計算理論についても復習するものとする。
\section{計算理論}
